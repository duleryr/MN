%%%%%%%%%%%%%%%%%%%%%%%%%%%%%%%%%%%%%%%%%%%%%%%%%%%%%%%
% 							                   PREAMBULE        
%%%%%%%%%%%%%%%%%%%%%%%%%%%%%%%%%%%%%%%%%%%%%%%%%%%%%%%%

\documentclass[a4,12pt]{article}

%--- Packages génériques ---%

\usepackage[francais]{babel}
\usepackage[utf8]{inputenc}
\usepackage[T1]{fontenc}
\usepackage[babel=true]{csquotes}
\usepackage{amsmath}
\usepackage{amssymb}
\usepackage{float}
\usepackage{graphicx}
\usepackage{hyperref}
\usepackage{xcolor}

%--- Structure de la page ---%

\usepackage{fancyheadings}

\topmargin -1.5 cm
\oddsidemargin -0.5 cm
\evensidemargin -0.5 cm
\textwidth 17 cm
\setlength{\headwidth}{\textwidth}
\textheight 24 cm
\pagestyle{fancy}
\lhead[\fancyplain{}{\thepage}]{\fancyplain{}{\sl ENSIMAG 1A}}
\chead[\fancyplain{}{{\sl }}]{\fancyplain{}{{TP Méthodes Numériques}}}
\rhead[\fancyplain{}{}]{\fancyplain{}{Polisano \& Jedouaa}}
\lfoot{\fancyplain{}{}}
\cfoot{\fancyplain{}{}}
\cfoot{\thepage }
\rfoot{\fancyplain{}{}}

%--- Style de la zone de code ---%

\usepackage{tikz}
\usetikzlibrary{calc}
\usepackage[framemethod=tikz]{mdframed}
\usepackage{listings}             
\usepackage{textcomp}

\lstset{upquote=true,
  columns=flexible,
  keepspaces=true,
  breaklines,
  breakindent=0pt,
  basicstyle=\ttfamily,
  commentstyle=\color[rgb]{0,0.6,0},
  language=Scilab,
  alsoletter=\),
}

\lstset{classoffset=0,
  keywordstyle=\color{violet!75},
  deletekeywords={zeros,disp},
  classoffset=1,
  keywordstyle=\color{cyan},
  morekeywords={zeros,disp},
}

\lstset{extendedchars=true,
  literate={0}{{\color{brown!75}0}}1 
  {1}{{\color{brown!75}1}}1 
  {2}{{\color{brown!75}2}}1 
  {3}{{\color{brown!75}3}}1 
  {4}{{\color{brown!75}4}}1 
  {5}{{\color{brown!75}5}}1 
  {6}{{\color{brown!75}6}}1 
  {7}{{\color{brown!75}7}}1 
  {8}{{\color{brown!75}8}}1 
  {9}{{\color{brown!75}9}}1 
  {(}{{\color{blue!50}(}}1 
  {)}{{\color{blue!50})}}1 
  {[}{{\color{blue!50}[}}1 
  {]}{{\color{blue!50}]}}1
  {-}{{\color{gray}-}}1
  {+}{{\color{gray}+}}1
  {=}{{\color{gray}=}}1
  {:}{{\color{orange!50!yellow}:}}1
  {é}{{\'e}}1 
  {è}{{\`e}}1 
  {à}{{\`a}}1 
  {ç}{{\c{c}}}1 
  {œ}{{\oe}}1 
  {ù}{{\`u}}1
  {É}{{\'E}}1 
  {È}{{\`E}}1 
  {À}{{\`A}}1 
  {Ç}{{\c{C}}}1 
  {Œ}{{\OE}}1 
  {Ê}{{\^E}}1
  {ê}{{\^e}}1 
  {î}{{\^i}}1 
  {ô}{{\^o}}1 
  {û}{{\^u}}1 
}

%--- Raccourcis commande ---%

\newcommand{\R}{\mathbb{R}}
\newcommand{\N}{\mathbb{N}}
\newcommand{\A}{\mathbf{A}}
\newcommand{\B}{\mathbf{B}}
\newcommand{\C}{\mathbf{C}}
\newcommand{\D}{\mathbf{D}}
\newcommand{\ub}{\mathbf{u}}
\newcommand{\redbox}[1]{\fcolorbox{red}{white}{\fcolorbox{red}{white}{$\displaystyle#1$}}}
 
%--- Mode correction et incréments automatiques ---%

\usepackage{framed}
\usepackage{ifthen}
\usepackage{comment}

\newcounter{Nbquestion}

\newcommand*\question{%
  \stepcounter{Nbquestion}%
  \textbf{Question \theNbquestion. }}

  \newboolean{enseignant}
%\setboolean{enseignant}{true}
  \setboolean{enseignant}{false}

  \definecolor{shadecolor}{gray}{0.80}

  \ifthenelse{
    \boolean{enseignant}}{
      \newenvironment{correction}{\begin{shaded}}{\end{shaded}}
    }
    {
      \excludecomment{correction}
    }

%--- Style de l'encadré des questions ---%

    \mdfsetup{leftmargin=12pt}
    \mdfsetup{skipabove=\topskip,skipbelow=\topskip}
    \tikzset{
      warningsymbol/.style={
	rectangle,draw=red,
	fill=white,scale=1,
      overlay}}
      \global\mdfdefinestyle{exampledefault}{
	hidealllines=true,leftline=true,
	innerrightmargin=0.0em,
	innerleftmargin=0.3em,
	leftmargin=0.0em,
	linecolor=red,
	backgroundcolor=orange!20,
	middlelinewidth=4pt,
	innertopmargin=\topskip,
      }

%%%%%%%%%%%%%%%%%%%%%%%%%%%%%%%%%%%%%%%%%%%%%%%%%%%%%%%%
% 							               EN-TETE        
%%%%%%%%%%%%%%%%%%%%%%%%%%%%%%%%%%%%%%%%%%%%%%%%%%%%%%%%

      \title{\textbf{TP Méthodes Numériques \\ Modélisation d'une corde de guitare et du tympan}}
      \author{
	\begin{tabular}{cc}
	  \textsc{Romain Duléry} & \textsc{Pierre Heinisch} \\
	\end{tabular}}   
	\date{\small \today}

	\makeatletter
	\def\thetitle{\@title}
	\def\theauthor{\@author}
	\def\thedate{\@date}
	\makeatother 

	\usepackage{etoolbox}
	\usepackage{titling}
	\setlength{\droptitle}{-7em}

	\setlength{\parindent}{0cm}

	\makeatletter
% patch pour le bug concernant les parenthèses fermantes d'après http://tex.stackexchange.com/q/69472
	\patchcmd{\lsthk@SelectCharTable}{%
	  \lst@ifbreaklines\lst@Def{`)}{\lst@breakProcessOther)}\fi}{}{}{}

%%%%%%%%%%%%%%%%%%%%%%%%%%%%%%%%%%%%%%%%%%%%%%%%%%%%%%%%
% 							CORPS DU DOCUMENT          
%%%%%%%%%%%%%%%%%%%%%%%%%%%%%%%%%%%%%%%%%%%%%%%%%%%%%%%%

	  \begin{document}
	  \maketitle

	  \begin{center}
	    \begin{tabular}{cc} 
	      \includegraphics[width=0.4\textwidth]{images/guitare.pdf} & 
	      \includegraphics[width=0.4\textwidth]{images/oreille.pdf} 
	    \end{tabular}
	  \end{center}
%%%%%%%%%%%%%%%%%%%%%%%%%%%%%%%%%%%%%%%%%%%%%%%%%%%%%%%%
% 						                  	PARTIE I         
%%%%%%%%%%%%%%%%%%%%%%%%%%%%%%%%%%%%%%%%%%%%%%%%%%%%%%%%

	  \section*{Partie I : Modélisation et simulation d'une corde de guitare}
	  \subsection*{1.1 : Modélisation physique}

	  La corde est régie par l'équation des ondes suivante :

	  \begin{equation}
	    \frac{\partial ^2u}{\partial t^2}=\gamma^2\frac{\partial ^2u}{\partial x^2}
	    \label{eq:ondes}
	  \end{equation}\\

	  \indent $\blacksquare$ \textbf{Modes propres de la vibration d'une corde}\\

%--- Question 1 ---%

	  \question On recherche une solution de la forme $u(x,t)=U(x)\cos(\omega t)$, on l'injecte donc dans l'équation (\ref{eq:ondes}), on obtient alors l'équation suivante :\\
	  \begin{center}
	    $-\omega^2  U(x)\cos(\omega t) = \gamma^2 U''(x) \cos(\omega t)$
	  \end{center}
	  Qui se simplifie en :
	  \begin{eqnarray}
	    U''(x) + \omega_0^2 U(x) = 0
	  \end{eqnarray}
	  avec $\omega_0 = \frac{\omega}{\gamma}$.\\

	  C'est une équation harmonique (à variable spatiale), on recherche donc une solution de la forme : $U(x) = A \cos(\omega_0 x) + B \sin(\omega_0 x)$\\

	  On utilise alors les conditions aux limites pour déterminer $A$ et $B$ :\\
	  \begin{itemize}
	    \item $u(0,t) = 0$  donc  $U(0) = 0$, ainsi $A = 0$.\\
	    \item $u(L,t) = 0$  donc  $U(L) = 0$, ainsi $B \sin(\omega_0 L) = 0$\\\\
	      C'est pourquoi :
	      $\exists n / \omega_0 = \frac{n\pi}{L} = \frac{\omega}{\gamma}$\\
	  \end{itemize}

	  On obtient donc une famille de solutions :
	  \[
	    U_n(x)=B_n\sin(n\pi \frac{x}{L}),\quad \omega_n=n\pi \frac{\gamma}{L}\Leftrightarrow f_n=n\frac{1}{2L}\sqrt{\frac{T}{\mu}}
	  \]

	  Il existe donc une infinité de modes propres de vibration de la corde, caractérisés par :

	  \[
	    u(x,t)=B_n\sin\left(n\pi \frac{x}{L}\right) \cos\left(n\pi \frac{\gamma t}{L}\right)
	  \]

	  Par principe de superposition, la solution générale de l'équation est de la forme :
	  \[
	    u(x,t)=\sum_{n=1}^{+\infty}B_n\sin\left(n\pi \frac{x}{L}\right) \cos\left(n\pi \frac{\gamma t}{L}\right)
	  \]

	  A l'instant $t = 0$, la solution devient :
	  \[
	    u(x,0)=\sum_{n=1}^{+\infty}B_n\sin\left(n\pi \frac{x}{L}\right)
	  \]

	  Ce qui exprime la convergence simple de la série de Fourier de la fonction $x \rightarrow u(x,0)$.\\
	  Par unicité des coefficients de Fourier, on a alors :
	  \[ 
	    \forall n \in \N, B_n = {2 \over L} \int_{0}^{L} u(x,0) \sin \left(n \pi {x \over L}\right) dx
	  \]

	  Analyse de la cohérence des résultats :\\\\
	  La fréquence fondamentale $f_1$ est inversement proportionnelle à la longueur de la corde. Ceci est cohérent car la note est d'autant plus grave que la corde est longue. De plus, lorsque la longueur de la corde est divisée par 2, la fréquence double, ce qui correspond à une octave.\\
	  Lorsque la masse linéique diminue, la fréquence augmente, donc la note est plus aigüe, ce qui est cohérent avec le fait que les cordes aïgues d'une guitare soient plus fines.\\
	  Enfin, lorsque la tension augmente, la fréquence également, ce que l'on peut remarquer lorsque on accorde une guitare en tendant plus ou moins les cordes.\\


%--- Fin Question 1 ---%

	  \indent $\blacksquare$ \textbf{Cas de la corde réelle avec raideur et amortissements}\\

	  La corde réelle est régie par l'équation aux dérivées partielles (EDP) suivante :

	  \begin{equation} 
	    \left\{
	      \begin{array}{l}
		\displaystyle \frac{\partial^2 u}{\partial t^2}=\gamma^2 \frac{\partial^2 u}{\partial x^2}-\kappa^2  				\frac{\partial^4 u}{\partial x^4}-2\sigma_0  \frac{\partial u}{\partial t}+2\sigma_1  \frac{\partial^3 u}{\partial t		\partial x^2}\\ \\ 
		\displaystyle u(0,t)=u(L,t)=\frac{\partial^2 u}{\partial x^2}(0,t)=\frac{\partial^2 u}{\partial x^2}(L,t)=0\\ \\  		\displaystyle u(x,0)=u_0(x), \quad \frac{\partial u}{\partial t}(x,0)=v_0(x)
	      \end{array}
	      \right. 
	      \label{eq:model_stiff}
	    \end{equation}

	    \subsection*{1.2 : Discrétisation du modèle par différences finies}

	    Nous allons utiliser la méthode des différences finies afin de rechercher une solution approchée de l'EDP : (\ref{eq:model_stiff}).\\\\
	    Discrétisons la corde de longueur $L=1$ en $N+1$ positions $x_l=lh$ avec $0\leqslant i\leqslant N$ et $h=\frac{1}{N}$, et le temps en $NF$ instants $t_n=nk$ avec $k=\frac{1}{SR}$ (où SR est le taux d'échantillonnage, typiquement 44100 Hz). Ainsi on approche la fonction continue $u(x,t)$ par $u_l^n$ en la position $x_l=lh$ et au temps $t_n=nk$, soit $u(x_l,t_n)=u(lh,nk)\approx u_l^n$. \\

	    Les deux prochaines questions consistent en l'obtention du schéma numérique.\\

%--- Question 2 ---%

	    \question Exprimons $u(x+h,t)$ et $u(x-h,t)$ sous forme de développement de Taylor à l'ordre 2 :

	    \[
	      u(x+h,t)=u(x,t)+h\frac{\partial u}{\partial x}(x,t)+\frac{h^2}{2!}\frac{\partial^2 u}{\partial x^2}(x,t)+o(h^2)
	    \]
	    \[
	      u(x-h,t)=u(x,t)-h\frac{\partial u}{\partial x}(x,t)+\frac{h^2}{2!}\frac{\partial^2 u}{\partial x^2}(x,t)+o(h^2)
	    \]

	    Ainsi,
	    \[
	      u(x+h,t)+u(x-h,t)=2u(x,t)+h^2 \frac{\partial^2 u}{\partial x^2}(x,t)+o(h^2)
	    \]

	    C'est pourquoi :

	    \begin{equation}
	      \frac{\partial^2 u}{\partial x^2}(x,t)=\frac{u(x+h,t)-2u(x,t)+u(x-h,t)}{h^2}+o(1)
	      \label{eq:taylor}
	    \end{equation}

	    De même, 
	    \begin{equation*}
	      \frac{\partial^2 u}{\partial t^2}(x,t)=\frac{u(x,t+k)-2u(x,t)+u(x,t-k)}{k^2}+o(1)
	    \end{equation*}\\

	    et donc, avec les notations introduites pour la discrétisation :
	    \[
	      \begin{array}{l}
		\displaystyle \frac{\partial^2 u}{\partial x^2}(x_l,t_k)\approx \frac{1}{h^2}(u_{l+1}^n-2u_l^n+u_{l-1}^n)\\ \\ 		\displaystyle \frac{\partial^2 u}{\partial t^2}(x_l,t_k)\approx \frac{1}{k^2}(u_{l}^{n+1}-2u_l^n+u_{l}^{n-1})\end{array}
	    \]

	    De même, en remarquant que : $$u(x+2h,t)+u(x-2h,t)-4[u(x+h,t)+u(x-h,t)]+6u(x,t) = \frac{\partial^4 u}{\partial x^4}(x,t) + o(h^4)$$ On obtient :
	    \[
	      \frac{\partial^4 u}{\partial x^4}(x_l,t_k)\approx \frac{1}{h^4}(u_{l+2}^n-4u_{l+1}^n+6u_l^n-4u_{l-1}^n+u_{l-2}^n)
	    \]

	    En dérivant chaque terme de la relation (\ref{eq:taylor}) grâce au fait que 
	    \[
	      \frac{\partial u}{\partial t}(x,t)=\frac{u(x,t+k)-u(x,t-k)}{2k}+o(1) \approx \frac{u_l^{n+1}-u_l^{n-1}}{2k}
	    \]
	    On obtient directement :
	    \[
	      \frac{\partial}{\partial t}\left(\frac{\partial^2 u}{\partial x^2}\right)\approx \frac{1}{2kh^2}(u_{l+1}^{n+1}-u_{l+1}^{n-1}-2u_{l}^{n+1}+2u_l^{n-1}+u_{l-1}^{n+1}-u_{l-1}^{n-1})
	    \]

	    Nous avons alors approché chaque terme de l'EDP (\ref{eq:model_stiff}) par des combinaisons linéaires de points de la grille de discrétisation, et en injectant les résultats obtenus, on obtient :
	    \begin{align}
	      &~~a_1u_{l-1}^{n+1}+a_2u_l^{n+1}+a_1u_{l+1}^{n+1}\\
	      +&~~b_1u_{l-2}^n+b_2u_{l-1}^n+b_3u_l^n+b_2u_{l+1}^n+b_1u_{l+2}^n\\ 
	      +&~~c_1u_{l-1}^{n-1}+c_2u_l^{n-1}+c_1u_{l+1}^{n-1}\\ 
	      =&~0
	    \end{align}

	    avec 
	    \[ 
	      a_1=-\frac{\sigma_1 k}{h^2},\quad a_2=1+\sigma_0 k+\frac{2\sigma_1 k}{h^2}, 
	      \quad c_1=\frac{\sigma_1 k}{h^2},\quad c_2=1-\sigma_0 k-\frac{2\sigma_1 k}{h^2}
	    \]
	    \[
	      b_1=\frac{\kappa^2 k^2}{h^4}, \quad b_2=-\frac{\gamma^2 k^2}{h^2}-\frac{4\kappa^2 k^2}{h^4},
	      \quad b_3=-2+\frac{2\gamma^2 k^2}{h^2}+\frac{6\kappa^2 k^2}{h^4}
	    \]
	    ~\\
%--- Fin Question 2 ---%

	    Nous nous intéressons désormais aux points du maillage situés à la frontière du domaine. On remarque que pour $l=1$ et $l=N-1$, on obtient respectivement $l-2=-1$ et $l+2=N+1$, on a donc besoin d'introduire des points dits \enquote{fantômes} ou \enquote{virtuels}, qu'on désigne par $u_{-1}^n$ et $u_{N+1}^n$.\\
	    ~\\
%--- Question 3 ---%

	    \question D'après la question précédente, on a :
	    \[
	      \frac{\partial^2 u}{\partial x^2}(x_l,t_k)\approx \frac{1}{h^2}(u_{l+1}^n-2u_l^n+u_{l-1}^n)
	    \]
	    Or les conditions aux limites imposent :
	    \[
	      \frac{\partial^2 u}{\partial x^2}(0,t)=\frac{\partial^2 u}{\partial x^2}(L,t)=0\
	    \]

	    Ainsi, on a :
	    \[
	      \frac{1}{h^2}(u_{1}^n-2u_0^n+u_{1}^n) \approx 0
	    \]
	    Or, la corde étant fixée aux extrémités, on a : $\forall n, u_0^n = u_N^n = 0$\\
	    On obtient alors $u_{-1}=-u_1$. De même, on obtient $u_{N+1}=-u_{N-1}$ à l'autre extrémité.\\

	    Par la suite on désignera par $\overline{\ub}^n$ le vecteur des points de discrétisation de la corde à $t=n$ 
	    \[
	      \overline{\ub}^n=\begin{bmatrix}u_0^n\\ u_1^n\\ \vdots \\ u_{N-1}^n \\ u_N^n\end{bmatrix}
	    \]

	    puis que le schéma numérique peut s'écrire sous forme matricielle
	    \begin{equation}
	      \overline{\A} \overline{\ub}^{n+1}+\overline{\B} \overline{\ub}^n+\overline{\C} \overline{\ub}^{n-1}=0
	    \end{equation} 
	    avec les matrices de tailles $(N+1)\times (N+1)$ suivantes : 
	    \[
	      \overline{\A}=
	      \begin{bmatrix}
		1 & 0 & 0 & 0 & \cdots & 0 \\ 
		a_1 & a_2 & a_1 & 0 & \cdots & 0 \\ 
		0 & a_1 & a_2 & a_1 & \ddots & 0 \\
		\vdots & \ddots & \ddots & \ddots & \ddots & \vdots \\ 
		0 & \cdots & 0 & a_1 & a_2 & a_1\\ 
		0 & \cdots & 0 & 0 & 0 & 1 \\
	      \end{bmatrix},\quad 
	      \overline{\C}=
	      \begin{bmatrix}
		1 & 0 & 0 & 0 & \cdots & 0 \\ 
		c_1 & c_2 & c_1 & 0 & \cdots & 0 \\
		0 & c_1 & c_2 & c_1 & \ddots & 0 \\ 
		\vdots & \ddots & \ddots & \ddots & \ddots & \vdots \\
		0 & \cdots & 0 & c_1 & c_2 & c_1 \\
		0 & \cdots & 0 & 0 & 0 & 1 \\
	      \end{bmatrix},
	    \]
	    \[ 
	      \overline{\B}=
	      \begin{bmatrix}
		1 & 0 & 0 & 0 & 0  & \cdots & 0 \\ 
		b_2 & b_3-b_1 & b_2 & b_1 & 0 & \cdots & 0 \\ 
		b_1 & b_2 & b_3 & b_2 & b_1 & \ddots & 0 \\
		\vdots & \ddots & \ddots & \ddots & \ddots & \ddots & \vdots \\ 
		0 & \cdots & b_1 & b_2 & b_3 & b_2 & b_1 \\ 
		0 & \cdots &  0 & b_1 & b_2 & b_3-b_1 & b_2 \\ 
		0 & \cdots & 0 & 0 &  0 & 0 & 1 \\
	      \end{bmatrix}
	    \]

%--- Fin Question 3 ---%

	    Notons que les $u_0^n$ et $u_{N}^n$ sont connus et valent zéro, on peut donc ne pas en tenir compte dans les calculs, ce qui revient à considérer le vecteur 
	    \[
	      \ub^n=\begin{bmatrix}u_1^n\\ \vdots \\ u_{N-1}^n\end{bmatrix}
	    \]
	    et les sous-matrices centrales de taille $(N-1)\times (N-1)$ : $\A$, $\B$ et $\C$ respectivement de $\overline{\A}$, $\overline{\B}$ et $\overline{\C}$, pour lesquelles on a supprimé la première et dernière ligne et colonne, à savoir :

	    \[
	      \A=
	      \begin{bmatrix} 
		a_2 & a_1 & 0 & \cdots & 0 \\ 
		a_1 & a_2 & a_1 & \ddots & \vdots \\ 
		0 & a_1 & a_2 & \ddots  & 0 \\ 
		\vdots & \ddots & \ddots & \ddots  & a_1 \\ 
		0 & \cdots & 0 & a_1 & a_2  \\
	      \end{bmatrix},\quad 
	      \C=
	      \begin{bmatrix} 
		c_2 & c_1 & 0 & \cdots & 0 \\ 
		c_1 & c_2 & c_1 & \ddots & \vdots \\ 
		0 & c_1 & c_2 & \ddots  & 0 \\ 
		\vdots & \ddots & \ddots & \ddots  & c_1 \\ 
		0 & \cdots & 0 & c_1 & c_2  \\
	      \end{bmatrix},
	    \]
	    \[
	      \B=
	      \begin{bmatrix}
		b_3-b_1 & b_2 & b_1 & 0 & \cdots & 0  \\ 
		b_2 & b_3 & b_2 & b_1 & \ddots & \vdots \\  
		b_1 & b_2 & b_3 & b_2 & \ddots & 0 \\ 
		0 & \ddots & \ddots & \ddots & \ddots &b_1\\ 
		\vdots & \ddots & b_1 & b_2 & b_3 & b_2  \\ 
		0 & \cdots &  0 & b_1 & b_2 & b_3-b_1  \\
	      \end{bmatrix}
	    \]

	    On définit la matrice Toeplitz suivante et son carré : 
	    \[
	      \D_{xx}=\frac{1}{h^2}
	      \begin{bmatrix}
		\ddots & \ddots & & & & \mathbf{0}  & \\
		\ddots & -2 & 1 & & & & \\ 
		& 1 & -2 & 1 & & & \\ 
		& & 1 & -2 & 1 & & \\ 
		& & & 1 & -2 & \ddots \\ 
		& \mathbf{0} & & & \ddots & \ddots
	      \end{bmatrix}, 
	      \quad \D_{xxxx}=\D_{xx}\D_{xx}
	    \]

%--- Question 4 ---%

	    \question Etant donné que 
	    \[
	      a_1=-\frac{\sigma_1 k}{h^2},\quad a_2=1+\sigma_0 k+\frac{2\sigma_1 k}{h^2}
	    \]
	    On a,
	    \[
	      \A=
	      \begin{bmatrix} 
		a_2 & a_1 & 0 & \cdots & 0 \\ 
		a_1 & a_2 & a_1 & \ddots & \vdots \\ 
		0 & a_1 & a_2 & \ddots  & 0 \\ 
		\vdots & \ddots & \ddots & \ddots  & a_1 \\ 
		0 & \cdots & 0 & a_1 & a_2  \\
	      \end{bmatrix} =
	      (1+\sigma_0 k)\mathbf{I} + \frac{\sigma_1 k}{h^2}
	      \begin{bmatrix}
		\ddots & \ddots & & & & \mathbf{0}  & \\
		\ddots & 2 & -1 & & & & \\ 
		& -1 & 2 & -1 & & & \\ 
		& & -1 & 2 & -1 & & \\ 
		& & & -1 & 2 & \ddots \\ 
		& \mathbf{0} & & & \ddots & \ddots
	      \end{bmatrix},
	    \]
	    Ainsi : \begin{equation}\A=(1+\sigma_0 k)\mathbf{I}-\sigma_1 k \D_{xx}\end{equation}

	    De même pour C :  \begin{equation}\C=(1-\sigma_0 k)\mathbf{I}+\sigma_1 k \D_{xx}\end{equation}

	    Quand à B, étant donné que
	    \[
	      b_1=\frac{\kappa^2 k^2}{h^4}, \quad b_2=-\frac{\gamma^2 k^2}{h^2}-\frac{4\kappa^2 k^2}{h^4},
	      \quad b_3=-2+\frac{2\gamma^2 k^2}{h^2}+\frac{6\kappa^2 k^2}{h^4}
	    \]
	    On a :
	    $$\B=-2\mathbf{I}-\gamma^2 k^2 \D_{xx} + \frac{\kappa^2 k^2}{h^4}
	    \begin{bmatrix}
	      5 & -4 & -1 & 0 & \cdots & 0  \\ 
	      -4 & 6 & -4 & 1 & \ddots & \vdots \\  
	      1 & -4 & 6 & -4 & \ddots & 0 \\ 
	      0 & \ddots & \ddots & \ddots & \ddots &1\\ 
	      \vdots & \ddots & 1 & -4 & 6 & -4  \\ 
	      0 & \cdots &  0 & 1 & -4 & 5  \\
	    \end{bmatrix}
	    $$

	    Et il se trouve que cette matrice est exactement la matrice $D_{xxxx}$, en prenant en compte le facteur $\frac{1}{h^4}$.\\

	    On obtient alors :
	    $$ \B=-2\mathbf{I}-\gamma^2 k^2 \D_{xx}+\kappa^2 k^2\D_{xxxx}, \quad $$

%--- Fin Question 4 ---%

	    \hspace{0.5cm} $\blacksquare$ \textbf{Stabilité du schéma numérique} \\

	    Par une analyse de Von Neumann on montre que la condition de stabilité du schéma s'écrit 
	    \[
	      h\geqslant h_{min}=\sqrt{\frac{\gamma^2k^2+\sqrt{\gamma^4 k^4+16\kappa^2 k^2}}{2}}
	    \]

	    \subsection*{1.3 : Programmation du schéma implicite sous Scilab}

	    \hspace{0.5cm} $\blacksquare$ \textbf{Choix des paramètres} $\sigma_0$  \textbf{et}  $\sigma_1$\\

%--- Question 5 ---%

	    \question On injecte $\tilde u(x,t)=e^{st+j\beta x}$ dans l'EDP (\ref{eq:model_stiff}) :

	    \[
	      s^2 = -\gamma^2 \beta^2 - \kappa^2 \beta^4 - 2 \sigma_0 s - 2 \sigma_1 \beta^2 s\\
	    \]
	    Soit :
	    \[
	      s^2 + 2(\sigma_0 + \sigma_1 \beta^2) s + \gamma^2 \beta^2 + \kappa^2 \beta^4 = 0
	    \]~\\
	    C'est une équation complexe du second degré, on calcule alors le discriminant réduit :
	    \[
	      \Delta' = (\sigma_0 + \sigma_1 \beta^2)^2 - \gamma^2 \beta^2 + \kappa^2 \beta^4
	    \]~\\
	    Et pour $\sigma_0\geqslant 0$ et $\sigma_1\geqslant 0$ petits, $\Delta' < 0$ donc les racines de l'équation caractéristique de l'EDP (\ref{eq:model_stiff}) sont données par $s_{\pm}=\sigma\pm j\omega$ où 
	    \[
	      \sigma(\beta)=-\sigma_0-\sigma_1\beta^2,\quad \omega(\beta)=\sqrt{\gamma^2 \beta^2+\kappa^2\beta^4-(\sigma_0+\sigma_1\beta^2)^2}
	    \]

	    Ainsi la perte $\sigma(\beta)$ décroît avec le nombre d'onde $\beta$, et vaut $-\sigma_0$ quand la raideur n'est pas prise en compte.\\

	    On peut alors inverser la fonction $\omega$ pour avoir l'expression de $\beta$ en fonction de $\omega$, afin d'exprimer la perte $\sigma$ en fonction de la fréquence $\omega$.

	    \[
	      \omega^2 = \gamma^2 \beta^2 + \kappa^2 \beta^4 - \sigma^2(\beta)
	    \]

	    En supposant $\sigma_0\geqslant 0$ et $\sigma_1\geqslant 0$ petits, on peut négliger $\sigma^2(\beta)$ dans cette expression.\\
	    On obtient alors, en notant $\xi = \beta^2$ :

	    \[
	      \kappa^2 \xi^2 + \gamma^2 \xi - \omega^2 = 0
	    \]
	    C'est une équation du second degré, on calcule donc le discriminant :

	    \[
	      \Delta = \gamma^4 + 4 \kappa^2 \omega^2 > 0
	    \]

	    Ainsi, on obtient l'expression de $\xi$ :

	    \[
	      \xi(\omega) = \frac{-\gamma^2+\sqrt{\gamma^4+4\kappa^2 \omega^2}}{2\kappa^2}
	    \]

	    C'est effectivement la seule solution que l'on retient, car $\xi = \beta^2$ force $\xi$ strictement positif.\\

	    Finalement, on obtient l'expression de $\sigma$ en fonction de la fréquence $\omega$ :

	    \[
	      \sigma(\omega)=-\sigma_0-\sigma_1\xi (\omega),\quad \xi(\omega)=\frac{-\gamma^2+\sqrt{\gamma^4+4\kappa^2 \omega^2}}{2\kappa^2}
	    \]

	    On note $T_{60}(\omega)$ la constante de décroissance dépendant de la fréquence $\omega$ et définie par 
	    \[
	      T_{60}(\omega)=-\frac{6\ln 10}{\sigma(\omega)}
	    \]
	    Soient 2 fréquences $\omega_1<\omega_2$, on a :

	    \[
	      \sigma(\omega_1)-\sigma(\omega_2) = -\sigma_1 \xi(\omega_1) + \sigma_1 \xi(\omega_2)
	      = -\frac{6 ln(10)}{T_{60}(\omega_1)} + \frac{6 ln(10)}{T_{60}(\omega_2)}
	    \]
	    Ainsi \[
	      \sigma_1=\frac{6 \ln 10}{\xi(\omega_2)-\xi(\omega_1)}\left(-\frac{1}{T_{60}(\omega_1)}+\frac{1}{T_{60}(\omega_2)}\right)
	    \]

	    De plus :
	    \[
	      \sigma(\omega_1)=-\frac{6\ln 10}{T_{60}(\omega_1)}
	    \]
	    Comme $\sigma(\omega)=-\sigma_0-\sigma_1\xi (\omega)$, on a :
	    \[
	      \sigma_0 = \frac{6 \ln 10}{\xi(\omega_2)-\xi(\omega_1)}\left(\frac{\xi(\omega_1)}{T_{60}(\omega_1)}-\frac{\xi(\omega_1)}{T_{60}(\omega_2)}\right) + \frac{6\ln 10}{T_{60}(\omega_1)}
	    \]
	    Ainsi :

	    \[
	      \sigma_0=\frac{6 \ln 10}{\xi(\omega_2)-\xi(\omega_1)}\left(\frac{\xi(\omega_2)}{T_{60}(\omega_1)}-\frac{\xi(\omega_1)}{T_{60}(\omega_2)}\right)
	    \]~\\

    %--- Fin Question 5 ---%

	    \hspace{0.5cm} $\blacksquare$ \textbf{Enregistrement stéréo et positionnement des micros}\\

    %--- Question 6 ---%

	    \question Ecrire un programme Scilab qui à partir des paramètres d'entrée fournis : \\

	    \begin{itemize}
	      \item[$\bullet$] Calcule itérativement le profil de la corde (vecteurs $\ub^n$) tout en affichant l'animation\\
	      \item[$\bullet$] Calcule et affiche le vecteur $out$ enregistrant les vibrations aux positions des micros $rp$\\ 	\item[$\bullet$] Emet le son de la corde en stereo à partir du vecteur $out$
	    \end{itemize}

	%--- Fin Question 6 ---%

	    \subsection*{1.4 : Analyses qualitative et quantitative du son produit}

	    On commence par s'assurer que la note simulée correspond bien à la fréquence fondamentale en entrée. Par exemple pour $f=110~Hz$, la note LA, vous pouvez vérifier avec une vraie guitare ou encore en décrochant votre téléphone (la tonalité est un LA par défaut). Mais on se propose ici de le vérifier mathématiquement (après tout votre instrument est peut-être mal accordé). 

	%--- Question 7 ---%

	    \question Dans cette question on attend de vous que vous preniez du recul quant à vos résultats obtenus.\\

	    \begin{itemize}
	      \item[$\bullet$] Calculer et afficher la transformée de Fourier du signal $s_1=out(:,1)$ via la commande $fft$ 	(la symétrie vous permet de ne garder qu'une partie du spectre, ou de repositionner correctement les 		fréquences autour de zéro via la commande $fftshift$). Déterminer l'indice auquel correspond le pic maximum 	du spectre. Cette fréquence fondamentale est bien celle que vous avez déclarée en entrée?\\ 
	      \item[$\bullet$] \textbf{Quantitativement} quel est l'effet sur le spectre du paramètre d'inharmonicité $B$? de 	la position $x_0$ où l'on pince la corde (vers le centre ou près de l'attache appelée le chevalet) ? \\ 
	      \item[$\bullet$] \textbf{Qualitativement} quels sont les effets de $B$ et $x_0$ sur le timbre du son joué?\end{itemize} 

	%--- Fin Question 7 ---%

	    \newpage

	%%%%%%%%%%%%%%%%%%%%%%%%%%%%%%%%%%%%%%%%%%%%%%%%%%%%%%%%
	% 						                  	PARTIE II         
	%%%%%%%%%%%%%%%%%%%%%%%%%%%%%%%%%%%%%%%%%%%%%%%%%%%%%%%%

	    \section*{Partie II : Modélisation de la membrane tympanique}

	    Dans cette partie, la propagation d'ondes en deux dimensions est illustrée par une vibration transverse d'une membrane élastique.

	    \subsection*{2.1 : Modélisation physique d'une membrane circulaire libre fixée sur les bords}

	    On cherche à résoudre le problème suivant, trouver $w(x,y,t)$ sur une durée $T_f$ solution de :
	    \begin{equation}
	      \left\{
		\begin{array}{rl}
		  \rho \dfrac{\partial^2 w}{\partial t^2} & =  T \Delta w ~\textrm{dans } ~ \Omega_T = \Omega \times (0,T_f)\\ 
		  w(x,y,t)& =  0 ~\textrm{sur} ~ \Gamma_T=\Gamma \times (0,T_f)\\
		  w(x,y,0)& = w_0(x,y) ~ \textrm{sur} ~\Omega \\
		  \dfrac{\partial w(x,y,0)}{\partial t} & = 0 ~\textrm{sur} ~\Omega 
		\end{array}
		\right.
		\label{eq:membranemodel}
	      \end{equation}

	      Soit $(x,y)$ les coordonnées cartésiennes d'un point de la membrane et $(r,\theta)$ ses coordonnées dans la base polaire on a :
	      \begin{center}
		\begin{minipage}[l]{.9\linewidth}

		  \begin{minipage}[l]{.4\linewidth}
		    \begin{equation*} 
		      \left\{ 
			\begin{array}{rcl}
			  x=r\cos(\theta) \\
			  y=r\sin(\theta)
			\end{array} 
			\right.
		      \end{equation*}
		    \end{minipage}
		    et
		    \begin{minipage}[r]{.4\linewidth}
		      \begin{equation*} 
			\left\{ 
			  \begin{array}{rcl}
			    r=\sqrt{x^2+y^2} \\
			    \theta=\arctan(\frac{y}{x})
			  \end{array} 
			  \right.
			\end{equation*}
		      \end{minipage}

		    \end{minipage}
		  \end{center}

		  avec $r \in[0,a]$ et $\theta  \in [0,2\pi]$.\\\\

    %--- Question 8 ---%

		  \question On cherche à passer le problème (\ref{eq:membranemodel}) en coordonnées polaires, étant donné que la membrane est circulaire :\\

		  On s'intéresse dans un premier temps à la première équation.\\

		  \begin{itemize} 
		    \item \textbf{Lemme :}\\

		    Soit f : $\R^2 \rightarrow \R$ une fonction régulière. Notons g la fonction définie sur $]0,+\infty[ \times [0,2\pi[$ par :
		      \[
			g(r,\theta) = f(x,y) = f(r \cos(\theta), r \sin(\theta))
		      \]

		      On a alors :

		      \begin{equation*}{}
			\left\{
			  \begin{array}{rl}
			    \dfrac{\partial f}{\partial x} = \cos \theta \dfrac{\partial g}{\partial r} - \dfrac{\sin \theta}{r} \dfrac{\partial g}{\partial \theta}\\\\
			    \dfrac{\partial f}{\partial y} = \sin \theta \dfrac{\partial g}{\partial r} + \dfrac{\cos \theta}{r} \dfrac{\partial g}{\partial \theta}\\
			  \end{array}
			  \right.
			\end{equation*}


			En effet, on a directement :

			\begin{equation*}{}
			  \left\{
			    \begin{array}{l}
			      \dfrac{\partial g}{\partial r}(r,\theta) = \cos \theta \dfrac{\partial f}{\partial x} + \sin \theta \dfrac{\partial f}{\partial y}\\\\
			      \dfrac{\partial g}{\partial \theta}(r,\theta) = - r \sin \theta \dfrac{\partial f}{\partial x} + r \cos \theta \dfrac{\partial f}{\partial y}\\
			    \end{array}
			    \right.
			  \end{equation*}

			  Ce qui permet d'obtenir le résultat voulu par combinaisons linéaires, en utilisant le fait que $\cos^2 + \sin^2 = 1$ \\

			\item \textbf{Expression du Laplacien bidimensionnel en coordonnées polaires :}\\

			  On note $g(r,\theta) = w(r \cos \theta, r \sin \theta)$

			  On a d'après le lemme :

			  \[
			    \dfrac{\partial^2 w}{\partial x^2} = \dfrac{\partial}{\partial x}\left( \dfrac{\partial w}{\partial x}\right) = \dfrac{\partial}{\partial x} \left( \cos \theta \dfrac{\partial g}{\partial r}(r,\theta) - \dfrac{\sin \theta}{r} \dfrac{\partial g}{\partial \theta}(r,\theta)  \right)
			  \]
			  D'après le théorème de Schwartz, cela devient :

			  \[
			    \dfrac{\partial^2 w}{\partial x^2} = \cos \theta \dfrac{\partial^2 w}{\partial r \partial x}(x,y) - \dfrac{\sin \theta}{r} \dfrac{\partial^2 w}{\partial \theta \partial x}(x,y)
			  \]

			  Or,

			  \begin{equation*}{}
			    \left\{
			      \begin{array}{l}
				\dfrac{\partial^2 w}{\partial r \partial x} = \dfrac{\partial w}{\partial r}\left(\dfrac{\partial w}{\partial x}\right) = \dfrac{\partial w}{\partial r}\left(\cos \theta \dfrac{\partial g}{\partial r} - \dfrac{\sin \theta}{r} \dfrac{\partial g}{\partial \theta}\right)\\\\
				\dfrac{\partial^2 w}{\partial \theta \partial x} = \dfrac{\partial w}{\partial \theta}\left(\dfrac{\partial w}{\partial x}\right) = \dfrac{\partial w}{\partial \theta}\left(\cos \theta \dfrac{\partial g}{\partial r} - \dfrac{\sin \theta}{r} \dfrac{\partial g}{\partial \theta} \right)
			      \end{array}
			      \right.
			    \end{equation*}

			    C'est à dire :
			    \begin{equation*}{}
			      \left\{
				\begin{array}{l}
				  \dfrac{\partial^2 w}{\partial r \partial x} = \cos \theta \dfrac{\partial^2 g}{\partial r^2}(r,\theta) + \dfrac{\sin \theta}{r} \dfrac{\partial g}{\partial r}(r,\theta) - \dfrac{\sin \theta}{r} \dfrac{\partial^2 g}{\partial r \partial \theta}(r,\theta)  \\\\
				  \dfrac{\partial^2 w}{\partial \theta \partial x} = - \sin \theta \dfrac{\partial g}{\partial r}(r,\theta) + \cos \theta \dfrac{\partial^2 g}{\partial r \partial \theta}(r,\theta) - \dfrac{\cos \theta}{r} \dfrac{\partial g}{\partial \theta}(r,\theta) - \dfrac{\sin \theta}{r} \dfrac{\partial^2 g}{\partial \theta^2}
				\end{array}
				\right.
			      \end{equation*}

			      Enfin, on a :

			      \[
				\dfrac{\partial^2 w}{\partial x^2} = \cos^2 \theta \dfrac{\partial^2 g}{\partial r^2}(r,\theta) + \dfrac{\sin^2 \theta}{r} \dfrac{\partial g}{\partial r}(r,\theta) + \dfrac{\sin^2 \theta}{r^2} \dfrac{\partial^2 g}{\partial \theta^2}(r,\theta)
			      \]

			      De même pour la dérivée seconde selon y :

			      \[
				\dfrac{\partial^2 w}{\partial y^2} = \sin^2 \theta \dfrac{\partial^2 g}{\partial r^2}(r,\theta) + \dfrac{\cos^2 \theta}{r} \dfrac{\partial g}{\partial r}(r,\theta) + \dfrac{\cos^2 \theta}{r^2} \dfrac{\partial^2 g}{\partial \theta^2}(r,\theta)
			      \]

			      Ainsi, en ajoutant ces deux égalités, on obtient l'expression du Laplacien :

			      \[
				\Delta w(r \cos \theta, r \sin \theta) = \dfrac{\partial^2 g}{\partial r^2}(r,\theta) + \dfrac{1}{r} \dfrac{\partial g}{\partial r}(r,\theta) + \dfrac{1}{r^2} \dfrac{\partial^2 g}{\partial \theta^2}(r,\theta)
			      \]


			    \item \textbf{Application au problème : }\\

			      L'EDP qui nous intéresse devient alors, sans distinguer la fonction de variables polaires et celle de variables cartésiennes :

			      \[
				\rho \dfrac{\partial^2 w}{\partial t^2} = \dfrac{\partial^2 w}{\partial r^2}(r,\theta) + \dfrac{1}{r} \dfrac{\partial w}{\partial r}(r,\theta) + \dfrac{1}{r^2} \dfrac{\partial^2 w}{\partial \theta^2}(r,\theta)
			      \]~\\

			      On introduit les variables adimensionnées suivantes 
			      \[
				\eta=\dfrac{r}{a},\tau=\dfrac{ct}{a}
			      \]
			      où $c=\sqrt{\frac{T}{\rho}}$ représente la vitesse de propagation  de l'onde.\\ 

			      On obtient alors :

			      \[
				\rho \dfrac{c^2}{a^2} \dfrac{\partial^2 w}{\partial \tau^2} = a^2  \dfrac{\partial^2 w}{\partial \eta^2}+\dfrac{a^2}{\eta}\dfrac{\partial w}{\partial \eta}+\dfrac{a^2}{\eta^2}\dfrac{\partial^2 w}{\partial \theta^2}
			      \]
			      ~\\
			      Et après simplifications, on obtient finalement :

			      \[\redbox{
				\dfrac{\partial^2 w}{\partial \tau^2} = \dfrac{\partial^2 w}{\partial \eta^2}+\dfrac{1}{\eta^2}\dfrac{\partial^2 w}{\partial \theta^2}+\dfrac{1}{\eta}\dfrac{\partial w}{\partial \eta},\;\eta\in[0,1],\;\theta\in[0,2\pi],\;\tau>0
			      }\]
			      ~\\

			      Passons à la seconde équation.\\

			      La membrane étant fixée sur le bord circulaire de rayon a, étant donné que l'on a posé $\eta=\dfrac{r}{a}$ unitaire, on a directement :

			      \[\redbox{
				w(1,\theta,\tau) = 0,\;\theta\in[0,2\pi],\;\tau\in[0,\frac{c}{a}T_f]
			      }\]
			      ~\\
			      La troisième équation, valable en tout point de la membrane circulaire, devient quant à elle :\\
			      \[\redbox{
				w(\eta,\theta,0) = w_0(\eta \cos(\theta),\eta \sin (\theta)),\;\eta\in[0,1],\;\theta\in[0,2\pi]
			      }\]
			      ~\\

			      Enfin, pour la dernière équation, étant sur $\Omega$, elle devient :

			      \[\redbox{
				\dfrac{\partial w(\eta,\theta,0)}{\partial t} = 0,\;\eta\in[0,1],\;\theta\in[0,2\pi]
			      }\]
			      ~\\
			      On ajoute à ces équations une condition limite nécessaire  en $\eta=0$ :
			      \[
				w(0,\theta,\tau)=w^0
			      \]

			  \end{itemize}

	%--- Fin Question 8 ---%

			  \subsection*{2.2 : Discrétisation polaire de la surface 2D par un schéma explicite }
			  Le but de cette partie est de discrétiser l'équation des ondes dans la base polaire (\ref{eq:polarmodel}) en utilisant un schéma explicite. Pour cela considérons dans un premier temps une discrétisation de $\Omega $ suivant une grille cartésienne $[0,N_x]\times [0,N_y]$ de pas uniforme $dx$, $dy$, formée par des points de coordonnées $(x_i,y_j)$ tel que :
			  \begin{equation*}{}
			    \left\{
			      \begin{array}{rl}
				x_i &=i dx,\quad 0\leq i \leq N_x \\
				y_j &=j dy,\quad  0\leq j \leq N_y \\
				dx & =\frac{1}{N_x+1}\\
				dy &=\frac{1}{N_y+1}\\
			      \end{array}
			      \right.
			    \end{equation*}
			    On note $dt$ le pas de temps et on approche la solution $w(x_i,y_j,ndt)$ par $w_{ij}^{n}$. Le laplacien est discrétisé  par un schéma centré d'ordre deux  et on utilise un schéma explicite d'ordre deux en temps. L'ordre de ces schémas peut être vérifié en utilisant un développement de Taylor. Le schéma explicite obtenue pour l'équation des ondes (\ref{eq:membranemodel}) est :

			    \begin{equation}
			      \left\{
				\begin{array}{rl}
				  \dfrac{w_{ij}^{n+1}-2w_{ij}^{n}+w_{ij}^{n-1}}{dt^{2}}=c^2\dfrac{w_{i+1,j}^{n}-2w_{i,j}^{n}+w_{i-1,j}^{n}}
				  {dx^{2}}+c^2\dfrac{w_{i,j+1}^{n}-2w_{i,j}^{n}+w_{i,j-1}^{n}}{dy^{2}}
				\end{array}
				\right.
				\label{eq:diffcartesiennes}
			      \end{equation}

%--- Question 9 ---%

			      \question Grâce à un développement de Taylor à
			      l'ordre 2, on
			      peut écrire :
			      \[
				w_{ij}^1 = w_{ij}^0 + dt\frac{\partial w_{ij}^0}{\partial t} + \frac{dt^2}{2}\frac{\partial^2 w_{ij}^0}{\partial t^2} + o(t^2)
			      \]

			      La vitesse à l'instant initial est nulle, on peut donc supprimer le
			      terme d'ordre 1. De plus le terme d'ordre 2 s'exprime également sous
			      forme de dérivées spatiales :

			      \[
				w_{ij}^1 = w_{ij}^0 +
				\frac{dt^2}{2}\frac{T}{\rho}\left(\frac{\partial^2 w_{ij}^0}{\partial x^2} +
				\frac{\partial^2 w_{ij}^0}{\partial y^2} \right)
			      \]

			      Or d'après les formules de Taylor :
			      \[
				\frac{\partial^2
				  w_{ij}^0}{\partial x^2} = \frac{1}{dx^2}(w_{i+1 j}^0 + w_{i-1 j}^0 -2w_{i j}^0)
				\]

				De même pour $\frac{\partial^2 w_{ij}^0}{\partial y^2}$, d'où finalement la relation suivante :
				\[
				  w_{ij}^1 = w_{ij}^0 + \frac{T}{\rho}\left(\frac{dt^2}{2dx^2}(w_{i+1 j}^0 + w_{i-1 j}^0 -2w_{i j}^0)+\frac{dt^2}{2dy^2}(w_{i j+1}^0 + w_{i j-1}^0 -2w_{i j}^0)\right) 
				\]


%--- Fin Question 9 ---%

%--- Question 10 ---%

				\question On note maintenant $w_{ij}^{n}=w(id\eta,jd\theta,ndt)$ les valeurs aux noeuds de la grille polaire représentée Figure \ref{fig:polaire}. Ecrire le schéma explicite de l'équation (\ref{eq:polarmodel}).

%--- Fin Question 10 ---%

				On montre qu'une condition limite en $\eta=0$ sur la grille polaire se détermine en utilisant le schéma explicite en coordonnées cartésiennes (\ref{eq:diffcartesiennes}), on obtient :
				\[
				  \dfrac{w_0^{n+1}-2 w_0^{n}+ w_{0}^{n-1}}{dt^2}=\dfrac{4}{d \eta^2}\left(\frac{1}{N_\theta} \sum_{j=1}^{N_\theta} w_{1,j}-w_{0}^{n}\right)
				\]

				\subsection*{2.3 : Stabilité et précision du schéma}

				\subsubsection{Etude de la stabilité}
				Commencons par une étude de la stabilité du schéma explicite (10), pour cela on remplace $w_{i,j}^{n}$ dans l'équation discrétisée, par une solution décomposée en modes de fourier: 
				\[
				  W_{p_{i,j}}^{n}=\phi^n e^{i\alpha_1(idx)}e^{i\alpha_2(jdy)}
				\]

    %--- Question 11 ---%

				\question Soit $J=\dfrac{\phi^{n+1}}{\phi^n}$ le coefficient d'amplification, montrer que $J$ vérifie l'équation du second degré suivante : 
				\[
				  J^2+\gamma J+1=0
				\]
				avec 
				\[
				  \gamma=-2+2c^2\dfrac{dt^2}{dx^2}(1-\cos(\alpha_1dx))+2c^2\dfrac{dt^2}{dy^2}(1-\cos(\alpha_2dy))
				\]
				En admettant que le schéma explicite (\ref{eq:membranemodel}) est stable sous la condition 
				\[
				  \vert J \vert \leq 1
				\]
				en déduire que cette condition de stabilité s'écrit aussi :
				\[
				  c^2\left(\dfrac{dt^2}{dx^2}+\dfrac{dt^2}{dy^2}\right)\leq 1
				\]

    %--- Fin Question 11 ---%

				Ainsi en posant $\displaystyle h=\dfrac{dxdy}{\sqrt{dx^2+dy^2}}$ on obtient la condition de courant : 
				\begin{equation}
				  \begin{array}{l}
				    CFL=c\dfrac{dt}{h} \leq 1~[\textrm{Cartésien}] \\ \\
				    \displaystyle CFL=\frac{d \tau}{d \eta d \theta}\leq 1~[\textrm{Polaire}]
				  \end{array}
				  \label{eq:CFL}
				\end{equation}

				\subsubsection{Etude de la consistance}
				On souhaite étudier la précision du schéma explicite (\ref{eq:diffcartesiennes}) en coordonnées cartésiennes.

	%--- Question 12 ---%

				\question En développant $w_{i,j}^{n+1},w_{i,j}^{n},w_{i,j}^{n-1}$ par un développement de Taylor montrer que le schéma est d'ordre 2 en temps \textit{i.e} l'erreur de troncature est en $o(dt^2)$.\\
				En suivant le même principe avec les termes en espace $w_{i+1,j}^{n},w_{i-1,j}^{n},w_{i,j-1}^{n},w_{i,j}^{n},w_{i,j+1}^{n}$ montrer que l'erreur de troncature est en $o (dx^2,dy^2)$.\\
				Ecrire alors l'erreur de troncature $E_t$ du schéma et en déduire que le schéma est d'ordre deux en temps et en espace. Ainsi le schéma explicite (\ref{eq:diffcartesiennes}) est consistant 
				\[
				  \lim_{(dt,dx,dy)\longrightarrow 0} E_t=0
				\]

	    %--- Fin Question 12 ---%

				\subsection*{2.4 : Solutions analytiques }
				On cherche une solution de notre équation d'onde polaire sous la forme :
				\[
				  w(\eta,\theta,\tau)=F(\eta,\theta)\cos(\omega \tau)
				\]
				avec $\omega$ la fréquence de vibration de l'onde et $F$ l'amplitude.

    %--- Question 13 ---%

				\question Ecrire l'équation différentielle vérifiée par l'amplitude $F$.\\

				La condition limite $F(1,0)=0$ nous permet de décomposer l'amplitude $F(\eta,\theta)$ sous la forme :
				\[
				  F(\eta,\theta)=\sum_n F_n(\eta)\cos(n\theta)
				\]
				Montrer que chaque $F_n$ vérifie l'équation de Bessel suivante, avec  $\alpha=\omega\eta$ :
				\[
				  \frac{d^2 F_n}{d\alpha^2}+\frac{1}{\alpha}\frac{dF_n}{d\alpha}+\left(1-\frac{n^2}{\alpha^2}\right)F_n=0
				\]

    %--- Fin Question 13 ---%

				\begin{table}
				  \begin{center}
				    \begin{tabular}{|c|c|c|c|c|c|}
				      \hline
				      $\lambda_{n,m}$&$m=1$  &$m=2$ & $m=3$ &$ m=4$&$m=5$\\
				      \hline
				      $n=0$&2.40483&$5.52008$&$8.65373$&$11.79153$&$14.93092$\\
				      \hline
				      $n=1$&$3.83171$&$7.01559$&$10.17347$&$13.32369$&$16.47063$\\
				      \hline
				      $n=2$&$5.13562$&$8.41724$&$11.61984$&$14.79595$&$17.95982$\\
				      \hline
				      $n=3$&$6.38016$&$9.76102$&$13.01520$&$16.22347$&$19.40942$\\
				      \hline
				      $n=4$&$7.58834$&$11.06471$&$14.37254$&$17.61597$&$20.82693$\\
				      \hline
				    \end{tabular}
				    \caption{Valeurs des cinq premiers zéros des fonctions de Bessel $J_n$.}
				  \end{center}
				\end{table}

				Les solutions de cette équation sont les fonctions de Bessel. Par ailleurs, c'est une équation du second ordre il existe donc deux solutions linéairement indépendantes. La solution est donc de la forme :
				\[
				  F_n(\eta)=d_nJ_n(\eta)+e_nY_n(\eta)
				\]
				avec les fonctions de Bessel d'ordre $n$ de première et deuxième espèce suivantes :
				\[
				  J_n(\eta)=\left(\dfrac{\eta}{2}\right)^n\sum_{m=0}^{\infty} \dfrac{(-\eta^2/4)^m}{m!\int_0^{\infty}e^{-t}t^{n+m}dt}
				\]
				\[
				  Y_n(\eta)=\dfrac{J_n(\eta)\cos(n\pi)-J_{-n}(\eta)}{\sin(n\pi)}
				\]
				Or les fonctions $Y_n(x)$ divergent au centre de la membrane $(\eta\rightarrow 0)$ donc $e_n=0$
				ainsi la solution analytique de l'équation (\ref{eq:polarmodel}) est donnée par :
				\[
				  w(\eta,\theta,\tau)=\cos(\omega \tau)\sum_n d_nJ_n(\omega\eta)\cos(n\theta)
				\]
				On sélectionne les fréquences en utilisant les conditions limites. Par exemple, dans le cas $n=0$, 
				\[
				  w(\eta,\theta,\tau)=\cos(\omega \tau) d_0J_0(\omega \eta)
				\]
				ainsi $\omega$ décrit les zéros de $J_0$ puisque la condition aux bords est $w(1,\theta,\tau)=0$. Si on note $\lambda_{m,0}$ la suite de ces zéros, on obtient les fréquences propres de vibration de la membrane :
				\[
				  \eta_{m,0}=\dfrac{\lambda_{m,0}}{2\pi a} \sqrt{\frac{T}{\rho}}
				\]
				En notant $\lambda_{m,n}$ la $m$-ième racine de $J_n$, on obtient de même 
				\[
				  \eta_{m,n}=\dfrac{\lambda_{m,n}}{2\pi a} \sqrt{\frac{T}{\rho}}
				\]

				Les valeurs numériques des racines des fonctions de Bessel $J_n$ sont fournies Table 1. Il en découle que les fréquences propres $\eta_{m,n}$ ne sont pas des multiples de la fréquence fondamentale. Par conséquent le son produit par la membrane n'est pas harmonique.

				\newpage 

				\subsection*{2.5 : Code scilab du schéma explicite et analyse des résultats}

				Afin de résoudre numériquement le schéma explicite en coordonnées polaires, on définit $w_{ij}^n$ comme un tableau de taille $(N_\eta)\times(N_\theta+1)$ avec :
				\[
				  \eta=id\eta ,\;i=1,..,N_\eta
				\]
				\[
				  \theta=jd\theta ,\;j=1,..,N_\theta+1
				\]
				On pose des conditions périodiques en $\theta$ :
				\[
				  w(\eta,\theta,\tau)=w(\eta,\theta+2\pi,\tau)
				\]

    %--- Question 14 ---%

				\question On prend la condition initiale suivante:
				\begin{equation}
				  w(\eta,\theta,0)=J_0(\lambda_{0,3}\eta)
				  \label{eq:condinit}
				\end{equation}
				avec $\lambda_{0,3}=8.65373$, la valeur approchée de la troisième valeur propre de $J_0$ correspondant à la fréquence du mode (0,3). De cette façon on a $d_0=1$.\\

				Ecrire un programme scilab qui résout numériquement le schéma explicite en coordonnées polaires de l'équation (\ref{eq:polarmodel}) avec les données suivantes : 
				\[
				  \begin{array}{ll}
				    \bullet & N_\theta=80 \\ 
				    \bullet & N_\eta=40 \\ 
				    \bullet & CFL=0.5 \\ 
				    \bullet & c=1
				  \end{array}
				\]
				Utilisez la fonction $besselj$ de Scilab pour implémenter la condition initiale (voir Figure \ref{fig:bessel}).\\
				Créez une animation représentant les vibrations de la membrane à différents instants $w^n$.

	%--- Fin Question 14 ---%

	%--- Question 15 ---%

				\question La solution analytique correspondant à la condition initiale (\ref{eq:condinit}) est :
				\[
				  w_{ex}(\eta,\theta,\tau)=\cos(\lambda_{0,3} \tau)J_0(\lambda_{0,3}\eta)
				\]
				Comparer visuellement les deux solutions (analytique $w_{ex}$ et numérique $w_{num}$) en vérifiant qu'elles possèdent bien le même comportement.
				Tracez l'erreur relative en $\eta=0$ en fonction du temps pour   différentes valeurs de la CFL. 
				Qu'observez-vous ?

    %--- Fin Question 15 ---%

    %--- Question 16 ---%

				\question On fixe la CFL à 0.8. Tracez l'erreur relative globale $\|w_{ex}-w_{num}\|_{\infty}$ pour les grilles de taille ($N_\theta=80$, $N_\eta=40$), ($N_\theta=40$,$N_\eta=20$) et ($N_\theta=160$,$N_\eta=80$).\\
				Comparez l'erreur relative obtenue pour les trois différentes grilles et en déduire que le schéma est d'ordre deux en espace.

	%--- Fin Question 16 ---%

	%--- Question 17 ---%

				\question On souhaite à présent représenter le mode (1,1) pour cela on introduit la condition initiale suivante:
				\[
				  w(\eta,\theta,0)=\dfrac{J_1(\lambda_{1,1}\eta)\cos(\theta)}{2}
				\]
				Créez une animation de la vibration de la membrane à différents instants.

				\newpage

				\end{document}

	% Fin du document LaTeX
